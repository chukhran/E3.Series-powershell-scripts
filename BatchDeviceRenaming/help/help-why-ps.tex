% language=uk

\startcomponent help-why-ps

\environment help-environment

\startsection[title=Why PowerShell]

For comfortable operation, the renaming process is required user interaction, in other words, GUI. There are a few solutions to get a GIU for Add-ons of \E\, however, for that need either Add-on should be compiled or new software should be installed. Using PowerShell solves the above problems: ps-scripts can have GUI without any compilation, PowerShell is also installed by default under any modern Windows OS.


\startsubsection[title=PowerShell scripts execution policy]

    In PowerShell, script execution is disabled by default. The goal of the execution policy mechanism is to reduce the ways that PowerShell can be exploited by an attacker, allowing a user to operate more securely.

    You can see the current execution policy by running \type{Get-ExecutionPolicy}:

    \starttyping
    PS> Get-Executionpolicy
    Restricted
    \stoptyping

    Here you see that the policy is set to \type{Restricted}.

    To modify it to run \type{BatchDeviceRenaming.ps1}, run \type{Set-ExecutionPolicy} in an elevated session and follow the prompts:

    \starttyping
    PS> Set-ExecutionPolicy RemoteSigned
    \stoptyping

    or (not recommended):

    \starttyping
    PS> Set-ExecutionPolicy Unrestricted
    \stoptyping

    \type{Set-ExecutionPolicy} command usually must be run by an administrator, because regular users don’t have permission to write to that portion of the Registry.

\stopsubsection

\stopsection

\stopcomponent
