% language=uk

\startcomponent help-gui-description

\environment help-environment

\startsection[title=Interface description]

\placefigure
[here]
[fig:main-window2]
{The main window of the script}
{\externalfigure[main-window.png][scale=800]}

\startsubsection[title=Device designation mask]

With this field, you can create a definition for a new device name. The buttons below allow inserting placeholders for the original name, parts of the original name, and a counter. Placeholders are always in brackets \type{[ ]}, while all other letters (without brackets) will be placed in the new name without a change.

Here is a description of all available placeholders:

\starttabulate[|rT|p|]
    \NC \color[maincolor]{[N]}      \NC An original device name; \NC \NR
    \NC \color[maincolor]{[N\#-\#]} \NC A part of the original name. \NC \NR
    \NC                             \NC Examples: \NC \NR
    \NC                             \NC \type{[N1-1]} -- The first character of the original name;\NC \NR
    \NC                             \NC \type{[N2-5]} -- Characters 2 to 5 from the original name (totals 4 characters). 
                                    The first letter is accessed with 1;\NC \NR
    \NC \color[maincolor]{[C]}      \NC A counter, as defined in the Define counter field. \NC \NR
\stoptabulate

\stopsubsection


\startsubsection[title={Define counter [C]}]
Allows defining the counter for the \type{[C]} placeholder(s).

\starttabulate[|r|p|]
    \NC \bf{Start at:}  \NC Number of the first device.; \NC \NR
    \NC \bf{Step by:}   \NC The counter is increased by this value.; \NC \NR
    \NC \bf{Digits:}    \NC Width of the counter field.
                            If the value is more than 1, the script will insert leading
                            zeros to get a fixed width number field. \NC \NR
\stoptabulate




\stopsubsection



\stopsection

\stopcomponent
